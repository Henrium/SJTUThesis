% !TEX root = ../thesis.tex

\chapter{Introduction}

Innovation in materials have pushed the frontier of the technology and society of humanity. Over centuries, however, research in materials science has followed an ``Edisonian'' methodology, which cannot keep up with the growing demand for advanced materials because of limited time and cost efficiency. With the development of experimental and computational techniques that enable generation of numerous data, and effective tools for data analysis, a data-driven paradigm  for materials research has emerged \parencite{MatDataSci}. Combined with automation systems, data-driven materials discovery can be achieved in an autonomous workflow \parencite{TBJoule}.

In this project, we develop a workflow for constructing complex phase diagrams using autonomous high-throughput experiment guided by active machine learning. This chapter first briefly introduces the background, followed by a review of methods for automation of data collection. Then, the present status and challenges in phase mapping are discussed. Finally, the problem and scope of our research project is stated, and the approach is proposed.

\section{Data-driven Materials Discovery}
Traditional materials development highly rely on scientific intuition and trial-and-error iteration. A typical procedure for developing a novel catalyst \parencite{Catalyst}, for example, includes design of chemical composition, synthesis, structural characterization, performance tests, and potentially repetition of the whole procedure with a modified design. The procedure is high in both economic and labor costs, and can be time-consuming. At the demand of efficient ways of materials innovation, the United States government launched Materials Genome Initiative (MGI), which emphasizes the combined force of experiment, computation, and informatics \parencite{FrontierMGI}. In the context of MGI, data holds a central position: high-throughput experiments and multiscale computational modeling are sources of data, whereas informatics tools provide means of data curation, mining, and analysis. 

One of the key tasks of materials research is to understand the relations between processing, structure, property, and performance, which can benefit significantly from data-driven approaches. Unveiling science from vast data dates back to the establishment of Kepler's Laws \parencite{4Para}; Hume-Rothery Rules stand as an example in materials science. As the scale and complexity of datasets grow far beyond human's capability of analysis, machine learning (ML) methods play an important part in extracting useful information from data. Researchers have utilized ML-related methods in the prediction of crystal stability \parencite{XtalStability}, the study on grain boundary evolution \parencite{gbdynamics}, and the analysis of experimental results, such as electron microscopic images \parencite{defect_in_EM} and electron diffraction patterns \parencite{EBSD_ML}.

Informatics tools also facilitate the design of materials, namely discovery of a tailored material with certain property or functionality as desired \parencite{InverseMolDesign}. In a recent study \parencite{MOFScreen}, the power of data-driven method is demonstrated in the design of \ce{CO2}-capturing metal--organic frameworks (MOFs): ML and first-principle calculations are performed to screen a large amount of candidates, followed by experiments in which the most promising candidates are synthesized and tested. This methodology is also applied in the computational design of light-emitting diodes (LEDs) \parencite{LEDScreen}, optoelectronic materials \parencite{Optoelectronic}, and piezoelectrics \parencite{Piezoelectrics}.

A critical point of data-driven materials research is the acquisition of data. Some of the practices use the data collected from previous experimental or computational works of the group \parencite{FailedExp}; others exploit open-access materials databases. Several databases have been developed by the materials community, among them the Inorganic Crystal Structure Database (ICSD) \parencite{ICSD} stands an early example. Modern materials data platforms curates data in a FAIR (findable, accessable, interoperatable, and reusable) fashion, and provide integrated module tools for data mining or visualization. Prominent examples include the Materials Project \parencite{MatProj}, which is comprehensive, and domain-specific platforms, such as NanoMine focusing on nanocomposites \parencite{NanoMine}. Published literature also supplies an abundant resource of data, from which useful information can be extracted by the means of text mining \parencite{CederTM,Zeolite}. Nevertheless, tracing to its source, the data is product of either experimentation or computation.

\section{Automation of Experiments}
Experiments and computations as sources of materials data can be time-consuming and high in both economic and labor costs. High-throughput experiments \parencite{CombiCuZn}, where up to thousands of samples are synthesized and characterized, and high-throughput computations \parencite{LiBatCalc}, where multiple input parameters are tested concurrently, provide improved production capacity of materials acquisition. However, in these tasks, major science facilities such as synchrotron radiation source, spallation neutron source, and supercomputers may be employed. These facilities are expensive and the availability is limited, thus making efficient design of experiments (DoE) highly desirable.

DoE in high-throughput experiment/computation contains a large decision space, \emph{i.e.} enormous choices of samples to characterize or parameters to use for calculation. The decision space can be narrowed down by setting criteria on the basis of prior knowledge or scientific intuition, but this kind of criteria are generally \emph{ad hoc}. Closed-loop workflow without human interference are preferable for attaining optimal DoE which enable efficient, autonomous experiments \parencite{AutoReview}. This can be realized by surrogate-based optimization or active learning (AL) approaches. In these approaches, predictive model is fitted based on results of previous queries, and the statistical inference made by the model is used as feedback to guide new queries. There have been efforts introducing automation into high-throughput materials investigation using the approaches mentionend above: Bayesian optimization has been applied by researchers in the design of materials with objective mechanical properties \parencite{bayesianAuto}; AL method has assisted in the computational discovery of electrocatalysts \parencite{ALCat}. And there remain various research areas and topics in materials science, where automation can be realized with the application of AL methods. The focus of this project is on one instance of such topics, the construction of ternary phase diagrams.

\section{Progress and Challenges in Phase Mapping}  % Demonstration
Phase diagrams, which describe the equillibrium phases coexisting under certain condition, provide reference and guidance for materials design. In particular, ternary isothermal phase diagrams illustrate mapping between composition and structure of ternary systems under fixed temperature and pressure. To simplify the nomenclature and make it easier to read, in this thesis the term ``phase diagram'' means ternary isothermal phase diagram unless otherwise stated.

In the conventional approach of constructing a phase diagram \parencite{CAWgamma}, obtaining one data point requires synthesizing sample with target composition, proper heat treatment, and characterization of microstructure using techniques such as X-ray diffraction (XRD) and scanning electron microscopy (SEM). This approach is exceptionally laborious; the heat treatment procedure can take thousands of hours. The emergence of combinatorial materials science brings time-efficient way of experimentally constructing phase diagrams in a high-throughput manner. By depositing thin-films with varying thickness on a substrate, a combinatorial materials chip covering the entire ternary composition space can be synthesized \parencite{FeCoNiACS}. Microstructure of the chip is then characterized pixel-by-pixel using microbeam XRD. From XRD pattern the phase corresponding to each composition can be identified and grouped, thereby the composition--phase linkage is established.

Besides experimental approach, thermodynamic calculations make another way of construcing phase diagrams \parencite{CAWcalphad}, also termed ``CALPHAD''. The phase diagrams derived solely from calculations are considered less reliable though, for lack of support from experimental facts. That being said, calculated phase distribution as prior knowledge can shed light on experimental investigations.

High-throughput characterization of combinatorial materials chips notably increases the efficacy of research on phase diagrams, yet there are potentials to further improve it. In order to attain high spatial resolution, the XRD measurement of combinatorial samples are often performed at a synchrotron radiation source \parencite{TiNiCu}. The characterization of one sample contains measurements of over 1,300 points, each taking no less than 15 \si{s} on the state-of-the-art facility or up to 2 \si{min} on less advanced counterparts. Typically several different settings are studied in one project, which requires several samples, so that the characterization experiments in one single project alone will hold the facility occupied for hours, even days. Provided that the number of measurements required to determine the phase distribution on one sample can be reduced, both higher efficiency and better instrument availability can be achieved. And this is feasible, 

\section{Objectives and Approach}  % Also: Organization

